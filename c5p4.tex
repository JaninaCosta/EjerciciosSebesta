\subsubsection{Pregunta 4: Subprogramas que definen variables que son utilizadas en otro subprograma}

\begin{verbatim}

def leer():
       listadeCompras=open('lista.txt','r')
       linea=listadeCompras.readline()
       while linea!="":
             linea=listadeCompras.readline()
             generarLista(linea)
             
def generarLista(linea):
                     if 'limon' or 'naranja' or 'kiwi' in linea:
                             lista1(linea)
                     elif 'almendras' or 'cacahuetes' or 'pimienta'  in linea:
                             lista2(linea)
                                      
def lista1(linea):
       print("Vitamina C:")
       for i in range(3):
             if i==1:
                    print(str(linea)+"sirve para los huesos")
             if i==2:
                    print(str(linea)+"sirve para los dientes")
             if i==3:
                    print(str(linea)+"reduce los sintomas de alergias")
               

def lista2(linea):
       print("Vitamina E:")
       for i in range(3):
             if i==1:
                    print(str(linea)+"sirve para la circulacion")
             if i==2:
                    print(str(linea)+"sirve para el cabello")
             if i==3:
                    print(str(linea)+"sirve para el colesterol")
                        
leer()

\end{verbatim}
