\subsubsection{Pregunta 6: Escriba programas de prueba en C++, Java, y C\# para determinar el alcance de una variable declarada en una sentencia for. Específicamente, el código debe determinar si esta variable es visible después del cuerpo de la sentencia for.}

\lstinputlisting[language=C++, frame=single,caption=Lenguaje C++]{c5p6.cpp}

\lstinputlisting[language=Java,frame=single, caption=Lenguaje Java]{c5p6.java}

\lstinputlisting[language=C,frame=single, caption= Lenguaje C \#]{c5p6.cs}

Podemos observar que en todos los tres lenguajes, el declarar una variable dentro del for y luego intentar accederla fuera del bloque for, nos genera un error. Esto se debe a que la variable tiene alcance local dentro del bloque for, y solo puede ser accedida dentro del mismo.
