\documentclass[12pt,oneside]{article}
\usepackage{geometry}                                % See geometry.pdf to learn the layout options. There are lots.
\usepackage{listings}				% Permite utilizar lenguajes de programacion dentro de latex
\geometry{a4paper}                                           % ... or a4paper or a5paper or ... 
%\geometry{landscape}                                % Activate for for rotated page geometry
%\usepackage[parfill]{parskip}                    % Activate to begin paragraphs with an empty line rather than an indent
\usepackage{graphicx}                                % Use pdf, png, jpg, or epsß with pdflatex; use eps in DVI mode
                                                                % TeX will automatically convert eps --> pdf in pdflatex                
\usepackage{amssymb}

\usepackage[spanish]{babel}                        % Permite que partes automáticas del documento aparezcan en castellano.
\usepackage[utf8]{inputenc}                        % Permite escribir tildes y otros caracteres directamente en el .tex
\usepackage[T1]{fontenc}                                % Asegura que el documento resultante use caracteres de una fuente apropiada.

\usepackage{hyperref}                                % Permite poner urls y links dentro del documento

\title{Ejercicios de Programación - Sebesta}
\author{Lenguajes de Programación - ESPOL}

%\date{}                                                        % Activate to display a given date or no date

\begin{document}
\maketitle

\section{Introducción}
Las respuestas propuestas en este repositorio son producto del trabajo de los estudiantes de la materia ``Lenguajes de Programación'' de la ESPOL, correspondientes a las preguntas del libro de Robert Sebesta, Concepts of Programming Languages.

\section{Preguntas y Respuestas}

\subsection{Capítulo 5: Nombres, Enlaces y Alcances.}
\subsubsection{Pregunta 4: Subprogramas que definen variables que son utilizadas en otro subprograma}

\begin{verbatim}

def leer():
       listadeCompras=open('lista.txt','r')
       linea=listadeCompras.readline()
       while linea!="":
             linea=listadeCompras.readline()
             generarLista(linea)
             
def generarLista(linea):
                     if 'limon' or 'naranja' or 'kiwi' in linea:
                             lista1(linea)
                     elif 'almendras' or 'cacahuetes' or 'pimienta'  in linea:
                             lista2(linea)
                                      
def lista1(linea):
       print("Vitamina C:")
       for i in range(3):
             if i==1:
                    print(str(linea)+"sirve para los huesos")
             if i==2:
                    print(str(linea)+"sirve para los dientes")
             if i==3:
                    print(str(linea)+"reduce los sintomas de alergias")
               

def lista2(linea):
       print("Vitamina E:")
       for i in range(3):
             if i==1:
                    print(str(linea)+"sirve para la circulacion")
             if i==2:
                    print(str(linea)+"sirve para el cabello")
             if i==3:
                    print(str(linea)+"sirve para el colesterol")
                        
leer()

\end{verbatim}

\subsubsection{Pregunta5: Generea la secuencia dada en C,C++ y java}

\begin{verbatim}

PROGRAMA EN C ---GENERA ERROR LA SECUENCIA PROPUESTA A COMPILAR

void main(void)
{
	x=21;
	printf("%i:",x);
	int x;
	x=42;
	printf("%i",x);
	Sleep(5000);
}

Lista de errores

Error	1	error C2065: 'x' : identificador no declarado	c:\users\dennise\documents\visual studio \\ 2010\projects\pruebadenisse\pruebadenisse.cpp	\\10	1	pruebaDenisse

Error	2	error C2065: 'x' : identificador no declarado	c:\users\dennise\documents\visual studio\\ 2010\projects\pruebadenisse\pruebadenisse.cpp	\\11	1	pruebaDenisse


PROGRAMA EN C++  ---GENERA ERROR LA SECUENCIA PROPUESTA A COMPILAR

void main(void)
{
	
	cout << "Ingresa 21:";
	cin >> x;
        int x;
	x=42;
	cout << "X vale:" << x;
	cin.get();cin.get();
	
}

Error	1	error C2065: 'x' : identificador no declarado \\	c:\users\dennise\documents\visual studio 2010\projects\pruebadenisse\pruebadenisse.cpp\\\	13	1	pruebaDenisse


PROGRAMA EN JAVA ---GENERA ERROR LA SECUENCIA PROPUESTA A COMPILAR

public class PruebaDenisse {

    static void main(String[] args) {
        x=21;
        int x;
        x=42;
       System.out.println(x);
    }
}

Exception in thread "main" java.lang.RuntimeException: \\ Uncompilable source code - cannot find symbol
  symbol:   variable x\\
  location: class pruebadenisse.NewMain \\
	at pruebadenisse.NewMain.main(NewMain.java:17)
Java Result: 1


\end{verbatim}

\subsection{Capítulo 9: SubProgramas.}
\subsubsection{Pregunta 4:Escribir en C\# un programa que pase por referencia un literal a un subprograma, que intenta cambiar el parametro. Dado la filosofia total de C\#, explique los resultados.}
\lstset{language = C}  %Seteo para poner Codigo de Lenguaje C
Código creado para la explicación en C\#.
\begin{lstlisting}[frame = single] %Comienzo del Código

 static void Main(string[] args)
        {
            String a;
            int b= -1;
            Program p = new Program();
            System.Console.Write("Ingrese numero:\n");
            a = Console.ReadLine();
            System.Console.Write("El numero es: " +a);
            System.Console.Write("\n");
            b = Convert.ToInt32(a);
            p.fun(ref b);
            System.Console.Write("El numero enviado a la funcion por referencia es: "+b);
            System.Console.Read();
            //System.Console.ReadKey();
        }

        public void fun (ref int x){
            x = x + 5;  
        }
\end{lstlisting}
OUTPUT:\\
	Ingrese numero: 6\\
	El numero es: 6\\
	El numero enviado a la funcion por referencia es:11\\
	El lenguague C\# recibe por defecto cuando se ingresa por teclado un Strign a este valor hay que convertirlo a un entero para enviarlo a nuestra funcion con la que queremos mostrar lo que deseamos.\\
 Lo que queremos mostrar es el cambio que se produce al valor de una variable siendo enviada a un subrprograma por referencia.\\


%\subsubsection{Pregunta 6: Escriba programas de prueba en C++, Java, y C\# para determinar el alcance de una variable declarada en una sentencia for. Específicamente, el código debe determinar si esta variable es visible después del cuerpo de la sentencia for.}

\lstinputlisting[language=C++, frame=single,caption=Lenguaje C++]{c5p6.cpp}

\lstinputlisting[language=Java,frame=single, caption=Lenguaje Java]{c5p6.java}

\lstinputlisting[language=C,frame=single, caption= Lenguaje C \#]{c5p6.cs}

Podemos observar que en todos los tres lenguajes, el declarar una variable dentro del for y luego intentar accederla fuera del bloque for, nos genera un error. Esto se debe a que la variable tiene alcance local dentro del bloque for, y solo puede ser accedida dentro del mismo.

%\subsubsection{Pregunta 7: Tres funciones en C donde se declare un arreglo de forma\\
estatica, otra stack y la ultima declaracion como heap}

\begin{verbatim}
STACK 
int main (int argc, char *argv[])
{
 
  int arreglo[3000];
  int i;
  srand(time(NULL));

  for(i=0;i<2999;i++)
  arreglo[i]=1+rand()%100;

  for(i=0;i<2999;i++){
  int doble;
  doble=arreglo[i]*2;
  printf("El doble del numero aleatorio en la posicion %d manejado por pila es: %d\n",i,doble);
 
  }
  return 0;
}

HEAP
int main (int argc, char *argv[])
{
 
  int arreglo[3000]; 
  int i;
  int *p;
  srand(time(NULL));
  


  for(i=0;i<2999;i++){
  p= (int *)malloc(3000*sizeof(int));
  arreglo[i]=1+rand()%100;
  *p=arreglo[i];
  }
 

  for(i=0;i<2999;i++){
  int doble;
  p= (int *)malloc(3000*sizeof(int));
  *p=arreglo[i];
  doble=*p*2;
  printf("El doble del numero aleatorio en la posicion %d manejado por heap es: %d\n",i,doble);
  
  }

  free(p);
  return 0;
}

STATIC

int main (int argc, char *argv[])
{
 
  static int arreglo[3000];
  int i;
  srand(time(NULL));

  for(i=0;i<2999;i++)
  arreglo[i]=1+rand()%100;

  for(i=0;i<2999;i++){
  int doble;
  doble=arreglo[i]*2;
  printf("El doble del numero aleatorio en la posicion %d con arreglo estatico es: %d\n",i,doble);
 
  }
  return 0;
}

\end{verbatim}
Stack tiene un aceso mas rapido, el espacio es manejado por el CPU es limitado y no puede ser redefinido.\\
En el caso del heap nosotros necesitamos manejar la memoria, acceso mas lento y no es limitado\\
Static declaracion unica de una variable que mantiene su dimension a lo largo del tiempo de vida del programa.



% Continuar con los siguientes capítulos y ejercicios:
% Ch6: 1, 2, 7
% Ch7: 1 - 6, 9
% Ch8: 3, 4, 5
% Ch9: 1, 5
% Recuerden que todos corresponden a las secciones de "Programming Exercises".

\end{document}
