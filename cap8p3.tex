\subsubsection{Capitulo 8 Pregunta 3 literal A:Rewrite the following code segment using a multiple-selection statement
in the following languages.}

\lstset{language = java}  %Seteo para poner Codigo de Lenguaje java
Código creado para la explicación en java.
\begin{lstlisting}[frame = single] %Comienzo del Código

public static void main(String[] args) {
        int k=0, j=0;
       printf("ingrese el valor de k");
        Scanner entrada =  new Scanner(System.in);
        k = entrada.nextInt();   
        
        switch(k){
            case 1:
              j=2*k-1;
	   printf("%d",j);
            case 2:
                j=2*k-1;
	     printf("%d",j);
            case 3:
                j=3*k+1;
               printf("%d",j);
            case 5:
                 j=3*k+1;
                printf("%d",j);
            case 4:
                j=4*k-1;
	     printf("%d",j);
            case 6:
                j=k-2;
	    printf("%d",j);
            case 7:
                j=k-2;
              printf("%d",j);
            case 8:
                 j=k-2;
	     printf("%d",j);
            default:
        }
               
    }
}

\end{lstlisting}
OUTPUT:\\
	Ingrese numero: 3\\
	El numero es: 3\\
	El valor de k es 3 y el programa de seleccion múltiple me lleva al case 5 \\
	con la respectiva función del case 5 la salida por pantalla será 16.\\
 


\lstset{language = python}  %Seteo para poner Codigo de Lenguaje Python
Código creado para la explicación en java.
\begin{lstlisting}[frame = single] %Comienzo del Código

#!/bin/env python
# -*- coding: utf8 -*-

print("ingrese un valor del 1 al 8 ")

def op1(k):
     j=0
     j=2*k-1
     return j

def op2(k):
     j=0
     j=3*k+1
     return j

def op3(k):
     j=0
     j=4*k-1
     return j
    
def op4(k):
    j=0
    j=k-2
    return j


k = raw_input("k: ")
 
operaciones = { '1': op1, '2': op1, '3': op2, '4': op3, '5' : op2 , '6' : op2,  '7' : op2, '8' : op2}
 
#seleccion = raw_input('Escoge una: ')

try:
    resultado = operaciones[k](int(k))
    print resultado
except:
    print("Esa opcion no existe")

\end{lstlisting}

OUTPUT:\\
	Ingrese numero: 3\\
	El numero es: 3\\
	El valor de k es 3 y el programa de seleccion múltiple me lleva al case 5 \\
	con la respectiva función del case 5 la salida por pantalla será 16.\\
	El try es solo para que se produzca una exception si el numero ingresado\\
	es diferente a las opciones disponibles, y el catch atrapa la excepcion\\
	haciendo los respectivos pasos \\
 




\lstset{language = c}  %Seteo para poner Codigo de Lenguaje C
Código creado para la explicación en java.
\begin{lstlisting}[frame = single] %Comienzo del Código


int main(){
     int k =0, j=0;
       scanf("%d",&k);
       switch(k){
               
            case 1:
              j=2*k-1;
              break;
            case 2:
                j=2*k-1;
                break;
            case 3:
                j=3*k+1;
                break;
            case 5:
                 j=3*k+1;
                 break;
            case 4:
                j=4*k-1;
                break;
            case 6:
                j=k-2;
                break;
            case 7:
                j=k-2;
                break;
            case 8:
                 j=k-2;
                 break;
            default:
                break;
       }
     
}

\end{lstlisting}

OUTPUT:\\
	Ingrese numero: 3\\
	El numero es: 3\\
	El valor de k es 3 y el programa de seleccion múltiple me lleva al case 5 \\
	con la respectiva función del case 5 la salida por pantalla será 16.\\
	
