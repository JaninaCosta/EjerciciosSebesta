\subsubsection{Capitulo 6 Pregunta 7 literal A:Write a C program that does a large number of references to elements
of two-dimensioned arrays, using only subscripting. Write a second
program that does the same operations but uses pointers and pointer
arithmetic for the storage-mapping function to do the array references.
Compare the time efficiency of the two programs. Which of the two
programs is likely to be more reliable? Why?.}

\lstset{language = java}  %Seteo para poner Codigo de Lenguaje java
Código creado para la explicación en java.
\begin{lstlisting}[frame = single] %Comienzo del Código

int main(){
    int i=0, j=0;
int matriz[30][5]=  {{2,4,0,3,4},{7,6,3,4,5},{9,6,3,4,5},
                    {3,1,8,2,11},{1,2,3,4,5},{4,1,3,4,7},
                    {1,7,1,0,2},{2,3,3,4,5},{0,7,7,4,5},
                    {9,6,3,4,5},{3,2,3,4,8},{2,4,0,3,4},
                    {9,6,3,4,5},{4,1,3,4,7},{1,7,3,60,5},
                    {10,9,3,4,7},{5,5,3,4,4},{0,6,3,8,5},
                    {3,6,30,30,5},{0,7,7,4,5},{9,3,1,4,3},
                    {5,60,3,4,5},{2,6,0,4,5},{3,2,3,4,8},
                    {1,7,3,60,5},{0,6,3,8,5},{0,7,7,4,5},
                    {9,6,3,0,5},{9,3,1,4,3},{0,7,7,4,5},};

for(i=0;i<20;i++){
    for(j=0;j<5;j++){
        printf("%d\t", matriz[i][j]);
}printf("\n");
}
getchar();
}


\end{lstlisting}


\lstset{language = python}  %Seteo para poner Codigo de Lenguaje Python
Código creado para la explicación en java.
\begin{lstlisting}[frame = single] %Comienzo del Código

int i,j, **p;
 int matriz[30][5]=  {{2,4,0,3,4},{7,6,3,4,5},{9,6,3,4,5},
                    {3,1,8,2,11},{1,2,3,4,5},{4,1,3,4,7},
                    {1,7,1,0,2},{2,3,3,4,5},{0,7,7,4,5},
                    {9,6,3,4,5},{3,2,3,4,8},{2,4,0,3,4},
                    {9,6,3,4,5},{4,1,3,4,7},{1,7,3,60,5},
                    {10,9,3,4,7},{5,5,3,4,4},{0,6,3,8,5},
                    {3,6,30,30,5},{0,7,7,4,5},{9,3,1,4,3},
                    {5,60,3,4,5},{2,6,0,4,5},{3,2,3,4,8},
                    {1,7,3,60,5},{0,6,3,8,5},{0,7,7,4,5},
                    {9,6,3,0,5},{9,3,1,4,3},{0,7,7,4,5},};
    p =matriz;
    for(i=0;i<30;i++)
         for(j=0;j<5;j++)              
             printf("matriz[%d][%d]=%d\n",i,j, *((p+i)+j));
    getchar();
    return 0
}

\end{lstlisting}
ANALISIS:\\
	Si se quiere optimizar el código se escribe con aritmética de punteros,\\
	si esto es trivial o predomina la legibilidad es preferible la indexación,\\
	estos no dejan de ser una abstracción que utilizan un simple vector de\\
	memoria contigua pero difieren en la  cuestion de tipos derivados como\\
	son, memoria automatica, asignacion dinamica, dobles punteros, arrays\\
	de punteros, arrays unidimensionales.
 



